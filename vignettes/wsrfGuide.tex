\documentclass[11pt,a4paper]{article}\usepackage{graphicx, color}
%% maxwidth is the original width if it is less than linewidth
%% otherwise use linewidth (to make sure the graphics do not exceed the margin)
\makeatletter
\def\maxwidth{ %
  \ifdim\Gin@nat@width>\linewidth
    \linewidth
  \else
    \Gin@nat@width
  \fi
}
\makeatother

\definecolor{fgcolor}{rgb}{0.2, 0.2, 0.2}
\newcommand{\hlnumber}[1]{\textcolor[rgb]{0,0,0}{#1}}%
\newcommand{\hlfunctioncall}[1]{\textcolor[rgb]{0.501960784313725,0,0.329411764705882}{\textbf{#1}}}%
\newcommand{\hlstring}[1]{\textcolor[rgb]{0.6,0.6,1}{#1}}%
\newcommand{\hlkeyword}[1]{\textcolor[rgb]{0,0,0}{\textbf{#1}}}%
\newcommand{\hlargument}[1]{\textcolor[rgb]{0.690196078431373,0.250980392156863,0.0196078431372549}{#1}}%
\newcommand{\hlcomment}[1]{\textcolor[rgb]{0.180392156862745,0.6,0.341176470588235}{#1}}%
\newcommand{\hlroxygencomment}[1]{\textcolor[rgb]{0.43921568627451,0.47843137254902,0.701960784313725}{#1}}%
\newcommand{\hlformalargs}[1]{\textcolor[rgb]{0.690196078431373,0.250980392156863,0.0196078431372549}{#1}}%
\newcommand{\hleqformalargs}[1]{\textcolor[rgb]{0.690196078431373,0.250980392156863,0.0196078431372549}{#1}}%
\newcommand{\hlassignement}[1]{\textcolor[rgb]{0,0,0}{\textbf{#1}}}%
\newcommand{\hlpackage}[1]{\textcolor[rgb]{0.588235294117647,0.709803921568627,0.145098039215686}{#1}}%
\newcommand{\hlslot}[1]{\textit{#1}}%
\newcommand{\hlsymbol}[1]{\textcolor[rgb]{0,0,0}{#1}}%
\newcommand{\hlprompt}[1]{\textcolor[rgb]{0.2,0.2,0.2}{#1}}%

\usepackage{framed}
\makeatletter
\newenvironment{kframe}{%
 \def\at@end@of@kframe{}%
 \ifinner\ifhmode%
  \def\at@end@of@kframe{\end{minipage}}%
  \begin{minipage}{\columnwidth}%
 \fi\fi%
 \def\FrameCommand##1{\hskip\@totalleftmargin \hskip-\fboxsep
 \colorbox{shadecolor}{##1}\hskip-\fboxsep
     % There is no \\@totalrightmargin, so:
     \hskip-\linewidth \hskip-\@totalleftmargin \hskip\columnwidth}%
 \MakeFramed {\advance\hsize-\width
   \@totalleftmargin\z@ \linewidth\hsize
   \@setminipage}}%
 {\par\unskip\endMakeFramed%
 \at@end@of@kframe}
\makeatother

\definecolor{shadecolor}{rgb}{.97, .97, .97}
\definecolor{messagecolor}{rgb}{0, 0, 0}
\definecolor{warningcolor}{rgb}{1, 0, 1}
\definecolor{errorcolor}{rgb}{1, 0, 0}
\newenvironment{knitrout}{}{} % an empty environment to be redefined in TeX

\usepackage{alltt}

%\VignetteEngine{knitr::knitr}




\usepackage[color=yellow]{todonotes}
\usepackage{graphicx}
\usepackage[T1]{fontenc}
\usepackage{hyperref}
\usepackage{natbib}
\definecolor{mediumseagreen}{rgb}{0.0255,0.4520,0.1394}
\hypersetup{colorlinks=true,
  filecolor=black,
  urlcolor=black,
  citecolor=mediumseagreen,
  linkcolor=blue,
  plainpages=false}

% \let\proglang=\textsf
% \newcommand\code{\bgroup\@makeother\_\@makeother\~\@makeother\$\@codex}
% \def\@codex#1{{\normalfont\ttfamily\hyphenchar\font=-1 #1}\egroup}
\newcommand{\proglang}[1]{\textsf{#1}}
\newcommand{\code}[1]{\texttt{#1}}
\newcommand{\Rdataset}[1]{\textit{#1}}
\newcommand{\pkg}[1]{{\fontseries{b}\selectfont #1}}

%% page layout
\topmargin 0pt
\textheight 46\baselineskip
\advance\textheight by \topskip
\oddsidemargin 0.1in
\evensidemargin 0.15in
\marginparwidth 1in
\oddsidemargin 0.125in
\evensidemargin 0.125in
\marginparwidth 0.75in
\textwidth 6.125in
\IfFileExists{upquote.sty}{\usepackage{upquote}}{}

\begin{document}


\title{Quick Start Guide for \pkg{wsrf}}
\author{He Zhao}
\maketitle


\section{Introduction}

The \proglang{R} package \pkg{wsrf} is a parallel implementation of
the Weighted Subspace Random Forest algorithm (wsrf) of
\cite{xu2012classifying}.  A novel variable weighting method is used
for variable subspace selection in place of the traditional approach
of random variable sampling.  This new approach is particularly useful
in building models for high dimensional data---often consisting of
thousands of variables.  Parallel computation is used to take
advantage of multi-core machines and clusters of machines to build
random forest models from high dimensional data with reduced elapsed
times.  The paper about its implementation and performance details is
under development.

\section{Requirements and Installation Notes}

Currently, \pkg{wsrf} requires R (>= 3.0.0), Rcpp (>= 0.10.2).  For
the use of multi-threading, a \proglang{C++} compiler with
\proglang{C++11} standard support of threads or the Boost
\proglang{C++} library \citep{Boost} with version above 1.54 is
required. The choice is available at installation time depending on
what is available to the user.  The development version of the
\pkg{wsrf} package is available at
\href{http://rattle.togaware.com/}{Togaware.com}.  To install the
latest development version of the package, from within \proglang{R}
run:

\begin{knitrout}
\definecolor{shadecolor}{rgb}{0.969, 0.969, 0.969}\color{fgcolor}\begin{kframe}
\begin{alltt}
\hlfunctioncall{install.packages}(\hlstring{"wsrf"}, repos=\hlstring{"http://rattle.togaware.com"})
\end{alltt}
\end{kframe}
\end{knitrout}


By default, the C++11 standard library is used for multi-threading,
other options can be configured through the argument
\code{configure.args}.

We recommend using C++11 standard library for accessing multi-threaded
functionality, which will be our main focus for development in the
future.  Though support for compiling C++11 code in packages is not
available in current release of \proglang{R}, it has been tested that
it can be compiled if the user has already installed the latest
version of \proglang{GCC} and \proglang{C++} standard
library\footnote{\proglang{C++11} support is experimental in R-devel
  now, see
  \url{http://developer.r-project.org/blosxom.cgi/R-devel/NEWS/2013/12/02\#n2013-12-02}}.

Besides the default installation for \proglang{C++11}, we also provide
another two implementations of \pkg{wsrf}.  One is an implementation
without parallelism where trees are built sequentially.  The other
implements parallelism using Boost.

The choice of version to install is available at build time.  To
install a version without parallelism, as required when C++11 is not
available nor is Boost, and is the recommended for Microsoft Windows
platform with the current version of R (3.0):
\begin{knitrout}
\definecolor{shadecolor}{rgb}{0.969, 0.969, 0.969}\color{fgcolor}\begin{kframe}
\begin{alltt}
\hlfunctioncall{install.packages}(\hlstring{"wsrf"}, 
                 repos=\hlstring{"http://rattle.togaware.com"},
                 type=\hlstring{"source"}, 
                 configure.args=\hlstring{"--enable-c11=no"})
\end{alltt}
\end{kframe}
\end{knitrout}

Finally the version using Boost for multithreading can be installed
with the appropriate configuration options. This is suitable when the
version of C++ available does not support C++11.
\begin{knitrout}
\definecolor{shadecolor}{rgb}{0.969, 0.969, 0.969}\color{fgcolor}\begin{kframe}
\begin{alltt}
\hlfunctioncall{install.packages}(\hlstring{"wsrf"}, 
                 repos=\hlstring{"http://rattle.togaware.com"},
                 configure.args="--with-boost-include=<Boost include path>
                                 --with-boost-lib=<Boost lib path>")
\end{alltt}
\end{kframe}
\end{knitrout}



\section{Usage} \label{sec:package}

This section demonstrates how to use \pkg{wsrf}, especially on a
cluster of machines.

The example uses a small dataset \Rdataset{weather} from \pkg{rattle}
\citep{rattle}.  See the help page of \pkg{rattle} in \proglang{R}
(\code{?weather}) for more details of \Rdataset{weather}.  Below are
the basic information of it.

\begin{knitrout}
\definecolor{shadecolor}{rgb}{0.969, 0.969, 0.969}\color{fgcolor}\begin{kframe}
\begin{alltt}
\hlfunctioncall{library}(rattle)
ds <- weather
\hlfunctioncall{dim}(ds)
\end{alltt}
\begin{verbatim}
## [1] 366  24
\end{verbatim}
\begin{alltt}
\hlfunctioncall{names}(ds)
\end{alltt}
\begin{verbatim}
##  [1] "Date"          "Location"      "MinTemp"       "MaxTemp"      
##  [5] "Rainfall"      "Evaporation"   "Sunshine"      "WindGustDir"  
##  [9] "WindGustSpeed" "WindDir9am"    "WindDir3pm"    "WindSpeed9am" 
## [13] "WindSpeed3pm"  "Humidity9am"   "Humidity3pm"   "Pressure9am"  
....
\end{verbatim}
\end{kframe}
\end{knitrout}


Before building the model we need to prepare the training dataset.
First we note the various roles played by the different variables,
including identifying the irrelevant variables

\begin{knitrout}
\definecolor{shadecolor}{rgb}{0.969, 0.969, 0.969}\color{fgcolor}\begin{kframe}
\begin{alltt}
target <- \hlstring{"RainTomorrow"}
id     <- \hlfunctioncall{c}(\hlstring{"Date"}, \hlstring{"Location"})
risk   <- \hlstring{"RISK_MM"}
ignore <- \hlfunctioncall{c}(id, \hlfunctioncall{if} (\hlfunctioncall{exists}(\hlstring{"risk"})) risk)

(vars <- \hlfunctioncall{setdiff}(\hlfunctioncall{names}(ds), ignore))
\end{alltt}
\begin{verbatim}
##  [1] "MinTemp"       "MaxTemp"       "Rainfall"      "Evaporation"  
##  [5] "Sunshine"      "WindGustDir"   "WindGustSpeed" "WindDir9am"   
##  [9] "WindDir3pm"    "WindSpeed9am"  "WindSpeed3pm"  "Humidity9am"  
## [13] "Humidity3pm"   "Pressure9am"   "Pressure3pm"   "Cloud9am"     
....
\end{verbatim}
\begin{alltt}
\hlfunctioncall{dim}(ds[vars])
\end{alltt}
\begin{verbatim}
## [1] 366  21
\end{verbatim}
\end{kframe}
\end{knitrout}


Next we deal with missing values, using \texttt{na.roughfix()} from
\pkg{randomForest} to take care of them.

\begin{knitrout}
\definecolor{shadecolor}{rgb}{0.969, 0.969, 0.969}\color{fgcolor}\begin{kframe}
\begin{alltt}
\hlfunctioncall{library}(randomForest)
\hlfunctioncall{if} (\hlfunctioncall{sum}(\hlfunctioncall{is.na}(ds[vars]))) ds[vars] <- \hlfunctioncall{na.roughfix}(ds[vars])
ds[target] <- \hlfunctioncall{as.factor}(ds[[target]])
(tt <- \hlfunctioncall{table}(ds[target]))
\end{alltt}
\begin{verbatim}
## 
##  No Yes 
## 300  66
\end{verbatim}
\end{kframe}
\end{knitrout}


We construct the formula that describes the model which will predict
the target based on all other variables.

\begin{knitrout}
\definecolor{shadecolor}{rgb}{0.969, 0.969, 0.969}\color{fgcolor}\begin{kframe}
\begin{alltt}
(form <- \hlfunctioncall{as.formula}(\hlfunctioncall{paste}(target, \hlstring{"~ ."})))
\end{alltt}
\begin{verbatim}
## RainTomorrow ~ .
\end{verbatim}
\end{kframe}
\end{knitrout}


Finally we create the randomly selected training and test datasets,
setting a seed so that the results can be exactly replicated.

\begin{knitrout}
\definecolor{shadecolor}{rgb}{0.969, 0.969, 0.969}\color{fgcolor}\begin{kframe}
\begin{alltt}
seed <- 42
\hlfunctioncall{set.seed}(seed)
\hlfunctioncall{length}(train <- \hlfunctioncall{sample}(\hlfunctioncall{nrow}(ds), 0.7*\hlfunctioncall{nrow}(ds)))
\end{alltt}
\begin{verbatim}
## [1] 256
\end{verbatim}
\begin{alltt}
\hlfunctioncall{length}(test <- \hlfunctioncall{setdiff}(\hlfunctioncall{seq_len}(\hlfunctioncall{nrow}(ds)), train))
\end{alltt}
\begin{verbatim}
## [1] 110
\end{verbatim}
\end{kframe}
\end{knitrout}


The signature of the function to build a weighted random forest model
in \pkg{wsrf} is:

\begin{knitrout}
\definecolor{shadecolor}{rgb}{0.969, 0.969, 0.969}\color{fgcolor}\begin{kframe}
\begin{alltt}
\hlfunctioncall{wsrf}(formula, 
     data, 
     ntrees=500, 
     nvars=NULL,
     weights=TRUE, 
     parallel=TRUE)
\end{alltt}
\end{kframe}
\end{knitrout}


We use the training dataset to build a random forest model.  All
parameters, except ``formula'' and ``data'', use their default values:
500 for ``ntrees'' --- the number of trees, the same as other package
(\pkg{randomForest} and \pkg{party}); TRUE for ``weights'' ---
weighted subspace random forest or random forest; TRUE for
``parallel'' --- use multi-thread or other options, etc.

\begin{knitrout}
\definecolor{shadecolor}{rgb}{0.969, 0.969, 0.969}\color{fgcolor}\begin{kframe}
\begin{alltt}
\hlfunctioncall{library}(wsrf)
model.wsrf <- \hlfunctioncall{wsrf}(form, data=ds[train, vars])
\hlfunctioncall{print}(model.wsrf, summary=TRUE)
\end{alltt}
\begin{verbatim}
## 
## Tree 1 with 51 nodes:
## 1) root
##  ..2) Pressure9am <= 1013.2
....
\end{verbatim}
\end{kframe}
\end{knitrout}


Here in the output, \textit{Strength} and \textit{Correlation} are two
measures introduced in \cite{breiman2001random} for evaluating a
random forest model.  \textit{Strength} measures the collective
performance of individual trees in a random forest and
\textit{Correlation} measures the diversity of the trees.

We can also obtain \textit{Strength} and \textit{Correlation} by:

\begin{knitrout}
\definecolor{shadecolor}{rgb}{0.969, 0.969, 0.969}\color{fgcolor}\begin{kframe}
\begin{alltt}
\hlfunctioncall{strength}(model.wsrf)
\end{alltt}
\begin{verbatim}
## strength 
##   0.6213
\end{verbatim}
\begin{alltt}
\hlfunctioncall{correlation}(model.wsrf)
\end{alltt}
\begin{verbatim}
## correlation 
##      0.1866
\end{verbatim}
\end{kframe}
\end{knitrout}


Then, \code{predict} the classes of test data.

\begin{knitrout}
\definecolor{shadecolor}{rgb}{0.969, 0.969, 0.969}\color{fgcolor}\begin{kframe}
\begin{alltt}
cl <- \hlfunctioncall{predict}(model.wsrf, newdata=ds[test, vars], type=\hlstring{"class"})
actual <- ds[test, target]
(accuracy.wsrf <- \hlfunctioncall{sum}(cl == actual, na.rm=TRUE)/\hlfunctioncall{length}(actual))
\end{alltt}
\begin{verbatim}
## [1] 0.8455
\end{verbatim}
\end{kframe}
\end{knitrout}


Thus, we have built a model that is 84.5455\% accurate on
unseen testing data.

To compare with \pkg{cforest} and \pkg{randomForest},

\begin{knitrout}
\definecolor{shadecolor}{rgb}{0.969, 0.969, 0.969}\color{fgcolor}\begin{kframe}
\begin{alltt}
\hlfunctioncall{library}(randomForest)
\hlfunctioncall{library}(party)
model.randomForest <- \hlfunctioncall{randomForest}(form, data=ds[train, vars])
model.cforest <- \hlfunctioncall{cforest}(form, data=ds[train, vars])

cl <- \hlfunctioncall{predict}(model.randomForest, newdata=ds[test, vars], type=\hlstring{"response"})
actual <- ds[test, target]
(accuracy.randomForest <- \hlfunctioncall{sum}(cl == actual, na.rm=TRUE)/\hlfunctioncall{length}(actual))
\end{alltt}
\begin{verbatim}
## [1] 0.8818
\end{verbatim}
\begin{alltt}

cl <- \hlfunctioncall{predict}(model.cforest, newdata=ds[test, vars], type=\hlstring{"response"})
actual <- ds[test, target]
(accuracy.cforest <- \hlfunctioncall{sum}(cl == actual, na.rm=TRUE)/\hlfunctioncall{length}(actual))
\end{alltt}
\begin{verbatim}
## [1] 0.8091
\end{verbatim}
\end{kframe}
\end{knitrout}


Next, we will specify building the model on a cluster of servers.

\begin{knitrout}
\definecolor{shadecolor}{rgb}{0.969, 0.969, 0.969}\color{fgcolor}\begin{kframe}
\begin{alltt}
servers <- \hlfunctioncall{paste0}(\hlstring{"node"}, 31:40)
model.wsrf <- \hlfunctioncall{wsrf}(form, data=ds[train, vars], parallel=servers)
\end{alltt}
\end{kframe}
\end{knitrout}


All we need is a character verctor specifying the hostnames of which
nodes to use, or a named integer vector, whose values of the elements
give how many threads to use for model building, in other words, how
many trees built simultaneously.  More detail descriptions about
\pkg{wsrf} are presented in.


\bibliographystyle{plain}
\bibliography{wsrfGuide}


\end{document}
